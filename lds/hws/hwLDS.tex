\documentclass{article}

\usepackage[hypertexnames=false,colorlinks=true]{hyperref}

\title{Linear Dynamical Systems homework}
\author{}
\date{}

\begin{document}

\maketitle

\section*{Problem 1: simulation of LDS}

Use the parameters in this
\href{https://github.com/joacorapela/probabilisticModelingCourse/blob/master/lds/data/00000002_simulation_params.npz}{file}
to sample N=10,000 states and observations from a LDS. Save the sampled values
to use them on the next problem. Make a scatter plot with the first and second
dimensions of the sampled states and obserfvations on the x- and y-axis,
respectively.

The parameters in the previous file were used to generate the corresponding
figure in the lectures notes. Hence your figure should be similar to the one on
these notes.

\section*{Problem 2: inference in an LDS}

Filter and smooth the simulated observations from the previous problem. The
following Python
\href{https://github.com/joacorapela/probabilisticModelingCourse/blob/master/lds/code/src/inference.py}{module}
provides incomplete code to perform Kalman filtering and smoothing. You may
want to complete this code to solve this problem.

Generate a first scatter plot as in the previous problem, showing the state
vertical and horizontal positions, the vertical and horizonal observation, and
the filtered and smoothed observations.

Generate a second scatter plot displaying the state, filtered and smoothed
vertical and horizontal velocities as a function of sample number. This plot
should contain six traces: \texttt{state\_vel\_x}, \texttt{filtered\_vel\_x},
\texttt{smoothed\_vel\_x}, \texttt{state\_vel\_y}, \texttt{filtered\_vel\_y} and
\texttt{smoothed\_vel\_y}.

Generate a third scatter plot displaying the state, filtered and smoothed
vertical and horizontal acelerations as a function of sample number.

\end{document}

