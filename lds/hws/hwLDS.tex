\documentclass{article}

\usepackage[hypertexnames=false,colorlinks=true]{hyperref}
\usepackage[shortlabels]{enumitem}

\title{Linear Dynamical Systems homework}
\author{}
\date{}

\begin{document}

\maketitle

\section*{Problem 1: simulation of an LDS}

\begin{enumerate}[(a)]

    \item Use the parameters in this
        \href{https://github.com/joacorapela/probabilisticModelingCourse/blob/master/lds/data/00000002_simulation_params.npz}{file}
        to sample N=10,000 states and observations from an LDS. Save the
        sampled values to use them on the next problem.

    \item Make a scatter plot with a trace with the first versus the second
        dimensions of the sampled states, and another trace with the first
        versus the second dimensions of the sampled observations.

\end{enumerate}

The parameters in the previous file were used to generate the corresponding
figure in the lectures notes. Hence your figure should be similar to the one on
these notes.

\section*{Problem 2: inference in an LDS}

\begin{enumerate}[(a)]

    \item Filter and smooth the simulated observations from the previous
        problem. The following Python
        \href{https://github.com/joacorapela/probabilisticModelingCourse/blob/master/lds/code/src/inference.py}{module}
        provides incomplete functions to perform Kalman filtering and
        smoothing. You may want to complete these functions to solve this
        problem.

    \item Generate a first scatter plot as in the previous problem, showing the
        state, observations, filtered and smoothed positions.

        Generate a second scatter plot displaying the state, filtered and
        smoothed velocities as a function of sample number. This plot should
        contain six traces: \texttt{state\_vel\_x}, \texttt{filtered\_vel\_x},
        \texttt{smoothed\_vel\_x}, \texttt{state\_vel\_y},
        \texttt{filtered\_vel\_y} and \texttt{smoothed\_vel\_y}.

        Generate a third scatter plot similar to the previous one, but for
        accelerations.

\end{enumerate}

\end{document}

